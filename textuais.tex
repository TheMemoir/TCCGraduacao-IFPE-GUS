\documentclass[a4paper,12pt,oneside,openright,extrafontsizes,openbib]{memoir}
%Define o tamanho da fonte para os título das seções primárias (Large)
\chapterstyle{section}
\renewcommand*{\chapnumfont}{\normalfont\Large\bfseries}
\renewcommand*{\chaptitlefont}{\normalfont\Large\bfseries}
%Define o cabeçalho para 0pt.
\setlength\beforechapskip{-\baselineskip}
%Algarismos arábicos nas páginas
\pagenumbering{arabic}
%Estilo de página
\pagestyle{simple}
%Define a PARTE (da classe Memoir) sem numeração de páginas.
\aliaspagestyle{part}{empty}
%Define secionamento numerado até a seção quaternária.
\settocdepth{subsubsection}
\setsecnumdepth{subsubsection}
%Config. das notas de rodapé.
\setlength{\footmarkwidth}{0em}
%Espaçamento de 1pt entre os parágrafos.
\setlength{\parskip}{1\baselineskip}
%Define a indentação padrão do parágrafos.
\setlength{\parindent}{1.25cm}
%Define o espaçamento entre linhas (exceto para citações longas).
\OnehalfSpacing
%========================================================
%INÍCIO
%PACOTES ESPECÍFICOS POR ÁREA DE ESTUDO/PESQUISA

%Inclua aqui outros pacotes necessários por área de estudo.
%Por exemplo: nenhum pacote para matemática foi includio 
%no modelo. Se for da área de exatas e for usar muitas equações,
%sugiro a inclusão dos pacotes para essa finalidade.


%PACOTES ESPECÍFICOS POR ÁREA DE ESTUDO/PESQUISA
%fim

%========================================================
%Geometria do tipo de papel (A4).
\usepackage[a4paper,%
		left=3cm,%
		right=2cm,% 
		top=3cm,% 
		bottom=2cm]{geometry}
%Medir margens do pacote Geometry.
\usepackage{calc}
%Fonte 10pt para legendas de figuras e tabelas
\usepackage[font=footnotesize]{caption}
%Gerar Lorem Ipsum.
\usepackage{lipsum}
%Idiomas (o último é o principal do documento).
\usepackage[spanish,english,brazil]{babel}
%Inserir acentos .
\usepackage[utf8]{inputenc}
%Tipo de fonte usada na compilação para incluir caracteres com acentos (Ö, é, à).
\usepackage[T1]{fontenc}
%Microtipografia conforme a língua utilizada.
\usepackage[babel=true]{microtype}
%Controle e personalização de listas
\usepackage{enumitem}
\setlist{nosep}
%Hiperlinks internos e externos.
%Esconder caixa vermelha das hiperligações no documento.
%Ativar links clicáveis no pdf, com cor azul.
\usepackage[breaklinks,%
			hidelinks,%
			colorlinks=true,%
			allcolors=blue]{hyperref}
%Digitar URL completas e clicáveis. 
%Caso queira links com a cor igual ao texto (preto), mudar acima para 'colorlinks=false'.
\usepackage{xurl}
%Para escrever códigos ou trechos de códigos. Ver ambiente CODEX.
\usepackage{verbatim}
%Para escrever versos. Funciona como o verbatim.
\usepackage{alltt}
%inserir imagens.
\usepackage{graphicx}
%Inserir páginas PDF.
\usepackage{pdfpages}
%Criar multicolunas sem o ambiente "tabular".
\usepackage{multicol}
%Caixas de alerta. Útil para livros didáticos ou destaque de algum trecho importante.
\usepackage{alertmessage}
%Cores.
\usepackage{xcolor}
%Diagramas.
\usepackage{smartdiagram}
%Tabelas mais fáceis.
\usepackage{tabularx}
%Mais controle e personalização em tabelas.
\usepackage{booktabs}
%Mudar espaçamento entre linhas.
\usepackage{setspace}
%Indentação de todos os primeiros parágrafos.
\usepackage{indentfirst}
%Digitar aspas com \qq e \q.
\usepackage{textcmds}
%Fonte padrão do documento.
\usepackage{times}
%Facilidades para manipualar formatação de FONTES (riscado, espaçamento entre carcateres...).
\usepackage{soul}
%Destaca a primeira letra do parágrafo. Fonte maior e em caixa de texto.
\usepackage{lettrine}
%Símbolos fonéticos.
\usepackage{tipa}
%CAIXAS DE TEXTO
%Caixas de texto "quebráveis entre páginas".
\usepackage{mdframed}
\usepackage{framed}
%TCOLORBOXES
\usepackage{tcolorbox}
\usepackage{varwidth}
\tcbuselibrary{most,listingsutf8}

%Referências e Citações
\usepackage[alf,
            abnt-emphasize=bf,%
            abnt-etal-list=3,%
            abnt-etal-text=emph,%
            abnt-missing-year=sd,%
            abnt-repeated-author-omit=yes,%
            abnt-repeated-title-omit=yes,%
            abnt-thesis-year=final,%
            abnt-doi=link,%
            ]{abntex2cite}

%AMBIENTES PERSONALIZADOS
%Verbatim modificaddo para incluir códigos em listas.
\newenvironment{mverbt}
{\verbatim}%
{\endverbatim}

%Para citações longas.
\newenvironment{citel}
{\SingleSpacing\footnotesize\list{}{\rightmargin=0cm \leftmargin=4cm}%
	\item\relax}%
{\endlist}

%Texto da folha de rosto (tipo de trabalho).
\newenvironment{textofolharosto}
{\SingleSpacing\small\list{}{\rightmargin=0cm \leftmargin=7cm}%
	\item\relax}%
{\endlist}

%Ambiente de caixa: CODEX.
\newtcblisting[auto counter,number within=chapter]{codex}[2][]{
	enhanced,
	listing only,
	title=Exemplo \thetcbcounter: #1 #2,
	listing options={style=tcblatex,%
		numbersep=1mm,%
		numbers=left,%
		numberstyle=\tiny\color{black}},
	before skip=2mm,after skip=2mm,
	colback=black!5,colframe=black!50,boxrule=0.2mm,
	attach boxed title to top left={xshift=1cm,yshift*=1mm-\tcboxedtitleheight},
	varwidth boxed title*=-3cm,
	boxed title style={frame code={
			\path[fill=tcbcolback!30!black]
			([yshift=-1mm,xshift=-1mm]frame.north west)
			arc[start angle=0,end angle=180,radius=1mm]
			([yshift=-1mm,xshift=1mm]frame.north east)
			arc[start angle=180,end angle=0,radius=1mm];
			\path[left color=tcbcolback!60!black,right color=tcbcolback!60!black,
			middle color=tcbcolback!80!black]
			([xshift=-2mm]frame.north west) -- ([xshift=2mm]frame.north east)
			[rounded corners=1mm]-- ([xshift=1mm,yshift=-1mm]frame.north east)
			-- (frame.south east) -- (frame.south west)
			-- ([xshift=-1mm,yshift=-1mm]frame.north west)
			[sharp corners]-- cycle;
		},interior engine=empty,
	},
	fonttitle=\bfseries,}
%Ambiente de caixa: OBSERVAÇÃO
\newtcolorbox{observ}[1][]{%
					arc=0mm,
					listing only,
					colback=black!5!white,
					colframe=red,
					fonttitle=\bfseries,
					title=Observação: #1,
}

%TCOLORBOX com molduras coloridas.
%Ambiente mvermelha
\colorlet{redbx}{red!75!black}
\newtcolorbox{mvermelha}[1]{%
	title={#1},
	empty,
	attach boxed title to top left,
	boxed title style={empty,size=minimal,
		toprule=2pt,top=4pt,
		overlay={\draw[redbx,line width=2pt]
			([yshift=-1pt]frame.north west)--([yshift=-1pt]frame.north east);}},
	coltitle=redbx,
	fonttitle=\Large\bfseries,
	before=\par\medskip\noindent,parbox=false,boxsep=0pt,left=0pt,right=3mm,top=4pt,
	breakable,pad at break*=0mm,vfill before first,
	overlay unbroken={\draw[redbx,line width=1pt]
		([yshift=-1pt]title.north east)--([xshift=-0.5pt,yshift=-1pt]title.north-|frame.east)
		--([xshift=-0.5pt]frame.south east)--(frame.south west); },
	overlay first={\draw[redbx,line width=1pt]
		([yshift=-1pt]title.north east)--([xshift=-0.5pt,yshift=-1pt]title.north-|frame.east)
		--([xshift=-0.5pt]frame.south east); },
	overlay middle={\draw[redbx,line width=1pt] ([xshift=-0.5pt]frame.north east)
		--([xshift=-0.5pt]frame.south east); },
	overlay last={\draw[redbx,line width=1pt] ([xshift=-0.5pt]frame.north east)
		--([xshift=-0.5pt]frame.south east)--(frame.south west);},%
}

%Ambiente mverde
\colorlet{greenbx}{green!75!black}
\newtcolorbox{mverde}[1]{%
	empty,title={#1},attach boxed title to top left,
	boxed title style={empty,size=minimal,toprule=2pt,top=4pt,
		overlay={\draw[greenbx,line width=2pt]
			([yshift=-1pt]frame.north west)--([yshift=-1pt]frame.north east);}},
	coltitle=greenbx,fonttitle=\Large\bfseries,
	before=\par\medskip\noindent,parbox=false,boxsep=0pt,left=0pt,right=3mm,top=4pt,
	breakable,pad at break*=0mm,vfill before first,
	overlay unbroken={\draw[greenbx,line width=1pt]
		([yshift=-1pt]title.north east)--([xshift=-0.5pt,yshift=-1pt]title.north-|frame.east)
		--([xshift=-0.5pt]frame.south east)--(frame.south west); },
	overlay first={\draw[greenbx,line width=1pt]
		([yshift=-1pt]title.north east)--([xshift=-0.5pt,yshift=-1pt]title.north-|frame.east)
		--([xshift=-0.5pt]frame.south east); },
	overlay middle={\draw[greenbx,line width=1pt] ([xshift=-0.5pt]frame.north east)
		--([xshift=-0.5pt]frame.south east); },
	overlay last={\draw[greenbx,line width=1pt] ([xshift=-0.5pt]frame.north east)
		--([xshift=-0.5pt]frame.south east)--(frame.south west);},%
}

%Caixas de textto com MDFRAMED
\newenvironment{bxpreta}
{\begin{mdframed}
		[skipabove=7pt,
		skipbelow=7pt,
		rightline=false,
		leftline=true,
		topline=false,
		bottomline=false,
		backgroundcolor=black!5,
		linecolor=black,
		innerleftmargin=5pt,
		innerrightmargin=5pt,
		innertopmargin=5pt,
		innerbottommargin=5pt,
		leftmargin=0cm,
		rightmargin=0cm,
		linewidth=5pt]}
	{\end{mdframed}}

\newenvironment{bxlaranja}
{\begin{mdframed}
		[skipabove=7pt,
		skipbelow=7pt,
		rightline=false,
		leftline=true,
		topline=false,
		bottomline=false,
		backgroundcolor=black!5,
		linecolor=orange,
		innerleftmargin=5pt,
		innerrightmargin=5pt,
		innertopmargin=5pt,
		innerbottommargin=5pt,
		leftmargin=0cm,
		rightmargin=0cm,
		linewidth=5pt]}
	{\end{mdframed}}

%Caixa Fundo cinza, Borda cinza
\newenvironment{bxcinza}
{\begin{mdframed}
		[skipabove=7pt,
		skipbelow=7pt,
		rightline=false,
		leftline=true,
		topline=false,
		bottomline=false,
		linecolor=gray,
		backgroundcolor=black!5,
		innerleftmargin=5pt,
		innerrightmargin=5pt,
		innertopmargin=5pt,
		leftmargin=0cm,
		rightmargin=0cm,
		linewidth=4pt,
		innerbottommargin=5pt]}
	{\end{mdframed}}

%Caixa Fundo azul, Borda azul
\newenvironment{bxazul}
{\begin{mdframed}
		[skipabove=7pt,
		skipbelow=7pt,
		rightline=false,
		leftline=true,
		topline=false,
		bottomline=false,
		linecolor=blue!40!gray,
		backgroundcolor=black!5,
		innerleftmargin=5pt,
		innerrightmargin=5pt,
		innertopmargin=5pt,
		leftmargin=0cm,
		rightmargin=0cm,
		linewidth=4pt,
		innerbottommargin=5pt]}
	{\end{mdframed}}

%Borda vermelha
\newenvironment{bxvermelha}
{\begin{mdframed}
		[skipabove=7pt,
		skipbelow=7pt,
		rightline=false,
		leftline=true,
		topline=false,
		bottomline=false,
		backgroundcolor=black!5,
		linecolor=red!40!gray,
		innerleftmargin=5pt,
		innerrightmargin=5pt,
		innertopmargin=5pt,
		innerbottommargin=5pt,
		leftmargin=0cm,
		rightmargin=0cm,
		linewidth=5pt]}
	{\end{mdframed}}

\begin{document}
%NÃO ALTERAR O TRECHO ABAIXO. SÃO AS CONFIGS. DOS PRÉ-TEXTUAIS.
    \begin{titlingpage}
	%PREENCHA OS CAMPOS ABAIXO COM AS INFORMAÇÕES DO SEU TRABALHO
%SUBSTITUA OS EXEMPLOS
\author{Leocádio Nestor Gumercindo Filho}
\title{A representação do arco-íris no Reino dos Unicórnios:\\ uma aplicação da variável CORES PRIMÁRIAS}
%Data da defesa. Apenas mês e ano.
\date{Jan. 2050}

%INFORMAÇÕES DA INSTITUIÇÃO E CURSO
\newcommand{\instituicao}{Instituto Federal de Pernambuco}
\newcommand{\campus}{Campus Garanhuns}
\newcommand{\curso}{Análise e desenvolvimento de sistemas}
\newcommand{\titulacao}{Tecnólogo}%Ou bacharel, ou licenciado.
\newcommand{\localdefesa}{Garanhuns - PE}
%Data exata da defesa para a folha de aprovação.
\newcommand{\dataexatadefesa}{01 de jan. \the\year}

%INFORMAÇÕES DO ORIENTADOR/A
\newcommand{\orientador}{Nome Completo do/a Orientador/a}
%Mude o texto da linha abaixo conforme o gênero do orientador/a.
\newcommand{\printorientador}{(Orientador/a)}
%Substitua "Titulação do orientador" por Esp., Me., Ma., Dr. ou Drª
\newcommand{\titorientador}{Dr.}
%Altere a linha abaixo caso seu orientador não seja do IFPE Garanhuns
\newcommand{\localorientador}{Instituto Federal de Pernambuco Campus Garanhuns}

%INFORMAÇÕES DOS AVALIADORES
%Avaliador/a 1
\newcommand{\avaliadorum}{Nome Completo do/a Avaliador/a 1}
\newcommand{\titavaliadorum}{Drª}% Esp., Me., Ma., Dr. ou Drª
\newcommand{\localavalum}{Universidade Floresta Mágica}% UFRPE, UFPE, UFMG etc por extenso
%Avaliador/a 2
\newcommand{\avaliadordois}{Nome Completo do/a Avaliador/a 1}
\newcommand{\titavaliadordois}{Me.}% Esp., Me., Ma., Dr. ou Drª
\newcommand{\localavaldois}{Instituto Bosque das Flores}% UFRPE, UFPE, UFMG etc por extenso

%FORMATAÇÃO DO ELEMENTOS PRÉ-TEXTUAIS - NÃO ALTERAR
%Capa
\begin{center}
%CAPA
%CASO NÃO DESEJE O LOGO DA INSTITUIÇÃO, EXCLUIR LINHA ABAIXO.
\includegraphics[scale=.10]{./img/logo-ifpe.png}\\
\textbf{\textsc{\instituicao}}\\
%Caso sua instituição não tenha CAMPUS, remova o comando \campus.
\textbf{\textsc{\campus}}\\
\textbf{\textsc{\curso}}


\vspace*{5cm}
\textbf{\thetitle}\\
\textbf{\theauthor}

\vspace*{\fill}
\localdefesa\\
\thedate
\end{center}

%FOLHA DE ROSTO + TIPO DE TRABALHO
\frontmatter{
\newpage
\thispagestyle{empty}
\begin{center}
    \textbf{\theauthor}
	
    \vspace*{5cm}
    \textbf{\thetitle}
\end{center}
%Caso sua instituição não tenha CAMPUS, remova o comando \campus.
%ATENÇÃO À PREPOSIÇÃO "AO/À"
\vspace*{2cm}
    \begin{textofolharosto}
    Trabalho de conclusão de curso apresentado ao/à \instituicao\ \campus\ como requisito parcial para obtenção do título de \titulacao\ em \curso.\\
    \ \\
    \textbf{Orientador:} \orientador
    \end{textofolharosto}

\begin{center}
\vspace*{\fill}
    \localdefesa\\
    \thedate
\end{center}

%FICHA CATALOGRÁFICA.
%Após a confecção da ficha catalográfica pela biblioteca de sua instituição --
%aqui, a biblioteca do IFPE Campus Garanhuns --
%imprima o arquivo (provavelmente em .docx) para pdf e salve-o
%na pasta pretxt do projeto com o nome "fichacatalografica.pdf".
\includepdf{fichacatalografica.pdf}

%FOLHA DE APROVAÇÃO
\newpage
\thispagestyle{empty}
\begin{center}
    \textbf{\textsc{\theauthor}}
	
    \textbf{\textsc{\thetitle}}
%ATENÇÃO AO USO DA PREPOSIÇÃO (DO/DA) ANTES DO COMANDO \instituicao	
\vspace*{2cm}
    \begin{textofolharosto}
        Este Trabalho de Conclusão de Curso foi julgado adequado para obtenção do Título de \titulacao\ e aprovado em sua forma final pelo Curso de \curso\ da \instituicao\ \campus.\\
        \ \\
        \ \\
        \localdefesa\ \-- \dataexatadefesa
        \end{textofolharosto}
	
        \vspace*{2cm}
        \line(1,0){10cm}\\
        \textbf{\titorientador\ \orientador\ \printorientador}\\
        \localorientador\\
        	
        \vspace*{2cm}
        \line(1,0){10cm}\\
        \textbf{\titavaliadorum\ \avaliadorum}\\
        \localavalum\\
        	
        \vspace*{2cm}
        \line(1,0){10cm}\\
        \textbf{\titavaliadordois\ \avaliadordois}\\
        \localavaldois
    \end{center}

%EPÍGRAFE
\newpage
\thispagestyle{empty}
\vspace*{\fill}
\epigraph{\textit{Uma frase importante relacionada ao trabalho.}}{\textsc{Quem disse a frase.}}

%DEDICATÓRIA
\newpage
\thispagestyle{empty}
\vspace*{15cm}
\begin{flushright}
    \SingleSpacing\textit{Dedico este trabalho ao ilustríssimo Professor Dr. André Padilha por ter me fornecido esse modelo de trabalho em \LaTeX\ e facilitar minha jornada na escrita acadêmica.}
\end{flushright}

%AGRADECIMENTOS
\newpage
\pagestyle{empty}
\begin{center}
	\textbf{\textsc{Agradecimentos}}
\end{center}
% Digite os agradecimentos a partir da linha abaixo 
% e SEMPRE após o comando no indent, após o primeiro
% afradecimento. EXEMPLO:
Agradeço incomensuravelmente ao ilustríssimo Professor Dr. André Padilha por ter me fornecido esse modelo de trabalho em \LaTeX\ e facilitar minha jornada na escrita acadêmica.

\noindent Abaixo, o resto dos agradecimentos...

\noindent Ao papai.

\noindent À mamãe.

\noindent Ao meu papagaio José Eustáquio. 

\noindent Ao \textit{Lorem ipsum}, abaixo.

\noindent  \lipsum[10]

%RESUMO e ABSTRACT
\newpage
\pagestyle{empty}
\begin{center}
    \textbf{\textsc{Resumo}}
\end{center}
\SingleSpacing\noindent
    O resumo de um trabalho acadêmico é normatizado pela ABNT 6028:2018. Deve ser escrito em 3ª pessoa, ter entre 150 a 500 palavras e apresentar: \textit{finalidades (objetivos), metodologia, referencial teórico, resultados e conclusões}. Deve ser escrito em bloco único, com frases concisas. Para trabalhos de graduação, a prática mais comum é ter até 300 palavras.

\vspace*{0.5cm}
\noindent\textbf{Palavras-chave:} Palavra 1; Palavra 2; Palavra 3. (Até cinco, separadas por \q{ ; } (ponto e vírgula).

%ABSTRACT
\newpage
\pagestyle{empty}
\begin{center}
    \textbf{\textsc{Abstract}}
\end{center}
\SingleSpacing\noindent
    Write your abstract here. Follow the same rules as indicated previously. Avoid using any automatic translation tool as \textit{Google Translator} except if you know what you are doing.

\vspace*{0.5cm}
\noindent\textbf{Keywords:} Word 1; Word 2; Word 3.
}
\newpage
\tableofcontents*
\thispagestyle{empty}
\newpage\listoffigures*
\thispagestyle{empty}
\newpage\listoftables*
\thispagestyle{empty}
    \end{titlingpage}
%INÍCIO DOS ELEMENTOS TEXTUAIS (CAPS./SEÇÕES)
\mainmatter{
\chapter*{Apresentação}\label{ch:apres}
\addcontentsline{toc}{chapter}{Apresentação}
Este modelo de documento foi elaborado para atender às necessidades dos estudantes dos cursos de graduação do IFPE Campus Garanhuns.

Quaisquer modificações que eventualmente venham a ser feita devem ocorrer no arquivo \textbf{definicoes.tex} e obedecer à estrutura do projeto. Veja o README.md no Github para explicações adicionais.

%Nome do seu capítulo/seção 1. Use a ref. cruzada como em \label
\chapter{Pacotes e ambiente incluídos}\label{ch:01}
Para detalhes sobre os pacotes incluídos nesse modelo, veja o arquivo \textbf{definicoes.tex} para informações básicas ou acesse os \textit{links} disponíveis ao longo do documento para ler a documentação. 

Os exemplos de uso, incluindo o código \LaTeX, de cada um deles estão na seção Apêndices.

\section{Lettrine}\label{sec:Lettrine}

\lettrine[findent=2pt]{\fbox{\textbf{V}}}{eja essa frase de abertura...} \lipsum[1]

\textsc{Pacote:} lettrine

Acesso: \url{https://www.ctan.org/pkg/lettrine}

\section{Verbatim}\label{sec:Verbatim}

\textsc{Pacote:} verbatim

Acesso: \url{https://www.ctan.org/pkg/verbatim}

\section{Inserir páginas em .pdf}\label{sec:pdfpages}

\textsc{Pacote:} pdfpages

\url{https://www.ctan.org/pkg/pdfpages}

\section{Texto em várias colunas}\label{sec:multicol}

\textsc{Pacote:} multicol

\url{https://www.ctan.org/pkg/multicol}

Outra possibilidade de inserção de texto em mais de uma coluna em uma página é através do uso de tabelas ou do ambiente \verb|minipage|. 

Para esse último, o código básico é:

\begin{codex}{Ambiente minipage básico}
    \begin{minipage}{4cm}
       Os {4cm} indicados acima apontam para a largura desejada da "minipage". 
    \end{minipage}
\end{codex}

Veja \url{https://www.alessandroduarte.com.br/?page_id=602} para um tutorial em português.

\section{Diagramas}\label{sec:smartdiagram{}

\textsc{Pacote:} smartdiagram

\url{https://www.ctan.org/pkg/smartdiagram}

\section{Aspas e outros símbolos tipográficos}\label{sec:textcmds}

\textsc{Pacote:} textcmds

\url{https://ctan.math.illinois.edu/macros/latex/contrib/amsrefs/textcmds.pdf}

Esse pacote foi incluído porque em modo \textit{offline} o \LaTeX habitualmente não reconhece os espaços necessários entre as aspas. O Overleaf já o faz nativamente (embora eu tenha observado erros nas aspas simples...).

\section{Hiperlinks com quebra de endereço por linha}\label{sec:xurl}

\textsc{Pacote:} xurl

\url{https://www.ctan.org/pkg/xurl}

\section{Tabelas}\label{sec:tabularx}

\textsc{Pacote:} tabularx

\url{https://www.ctan.org/pkg/tabularx}

\section{Imagens}\label{sec:graphicx}

\textsc{Pacote:} graphicx

\url{https://www.ctan.org/pkg/graphicx}

\section{Inserção de códigos de programação}\label{sec:codes}

\textsc{Importante:} Somente consegui reproduzir no Overleaf o pacote \verb|codebox|. Os pacotes \verb|shdoc| e \verb|pygmentex| não compilaram adequadamente, resultando em erros. Sugiro que tentem no modo \textit{offline} (\LaTeX\ instalado no computador).

\subsection{codebox}\label{subsec:codebox}
\textsc{Pacote:} codebox

\url{https://www.ctan.org/pkg/codebox}

\subsection{shdoc}\label{subsec:shdoc}
\textsc{Pacote:} shdoc

\url{https://www.ctan.org/pkg/shdoc}

\subsection{pygmentex}\label{subsec:pygmentex}
\textsc{Pacote:} pygmentex

\url{https://www.ctan.org/pkg/pygmentex}

\section{Ambientes \LaTeX\ personalizados}\label{sec:personal}

\subsection{codex}\label{subsec:codex}

\subsection{observ}\label{subsec:observ}

\subsection{citel}\label{subsec:citel}

\subsection{Caixas de Texto}\label{subsec:caixatextos}

Há diversos ambientes possíveis para criar caixas de texto neste modelo. São eles:

\begin{enumerate}
    \item codex
    \item observ
    \item mverde
    \item mvermelha
    \item bxpreta
    \item bxlaranja
    \item bxcinza
    \item bxazul
    \item bxvermelha
    \item \textbf{alertmessage\footnote{Exceção aos listados acima. Trata-se de um comando, não de um ambiente.}}
\end{enumerate}

\textsc{Pacotes:}
\begin{enumerate}
    \item tcolorbox
    \begin{itemize}
        \item Acesso: \url{https://www.ctan.org/pkg/tcolorbox}
    \end{itemize}
    \item framed
    \begin{itemize}
        \item Acesso: \url{https://www.ctan.org/pkg/framed}
    \end{itemize}
    \item mdframed
    \begin{itemize}
        \item Acesso: \url{https://www.ctan.org/pkg/mdframed}
    \end{itemize}
    \item alertmessage
    \begin{itemize}
        \item Acesso: \url{https://www.ctan.org/pkg/alertmessage}
    \end{itemize}
\end{enumerate}

\chapter{Estrutura do projeto}

Esse modelo contém seis arquivos e uma pasta que devem ser enviados para o Overleaf após a criação da conta gratuita na plataforma.

No caso de uso \textit{offline} do \LaTeX, descompacte o .zip baixado do GitHub para uma pasta e siga as mesmas orientações.

\section{definicoes.tex}

Contém todas as definições do modelo conforme a ABNT 14721:2011 além de todos os pacotes e ambientes utilizados. Ver os \textbf{Apêndices} para uma explicação dos comandos e ambientes.

\section{pretextuais.tex}

Estão todas as informações necessárias ao projeto, tais como:

\begin{itemize}
    \item Capa
    \item Folha de rosto
    \item Ficha catalográfica (ver explicação abaixo)
    \item Folha de aprovação
    \item Epígrafe
    \item Dedicatória
    \item Agradecimentos
    \item Resumo e Abstract
\end{itemize}

É nesse arquivo que as informações do estudante, título do trabalho, orientador, avaliadores, etc estão incluídas. \textsc{LEIA ESSE ARQUIVO COM ATENÇÃO.}

\section{fichacatalografica.pdf}

Normalmente, a ficha catalográfica é feita por um/uma bibliotecário/a e entregue em formato .doc, .docx ou .odt. 

Após a elaboração pelo/a profissional responsável, imprima/converta esse arquivo para .pdf e substitua o arquivo \verb|fichacatalografica.pdf| que já vem no projeto.

\section{textuais.tex}

É o arquivo onde se digitam as seções (capítulos) do seu trabalho. Atentar para a primeira seção (Apresentação). A ABNT 14724:2011 não informa que esta deva ser não-numerada. Por prática, o projeto apresenta essa seção não numerada e incluída já no \textbf{Sumário}.

\section{referencias.bib}

Contém exemplos em branco para auxiliar o preenchimento das referências bem como exemplos já preenchidos para servirem de modelos.

Alguns exemplos de citação vazia seguem abaixo.

\cite{Smurf1980} - Citação direta de artigo, sem fazer parte do texto.

\cite[p. 654]{Vingadores2015} - citação direta de livro com página, sem fazer parte do texto.

\citeonline{Flintstones1900} - citação direta de um livro com Organizador (Org.), sem fazer parte do texto.

\citeonline[p. 89]{Flintstones1900} - a mesma citação acima, com indicação da página.

\cite{Vingadores2016} - citação direta de uma dissertação de mestrado, sem fazer parte do texto. (Ver próxima citação e referência)

\cite{Thor2017} - citação direta de uma tese de doutorado, sem fazer parte do texto e com mesmo autor da citação acima. Observar as \textbf{Referências}.

\section{postextuais.tex}

Contém os Anexos e os Apêndices. Estes são elementos opcionais e, neste modelo, na parte específica dos Apêndices, trazem os exemplos de códigos, pacotes e/ou ambientes utilizados.

\section{pasta img}

Contém as imagens utilizadas no projeto. Use o formato .png e .jpg para não precisar instalar pacotes extras ou \qq{quebrar muito a cabeça} com conversões entre formatos.

\chapter{Referências e Citações}\label{ch:refs-cit}

Para a utilização correta do pacote \verb|abntex2cite| nesse projeto, é essencial a leitura dos seguintes materiais disponíveis em \url{https://www.ctan.org/pkg/abntex2}, em especial:

\begin{itemize}
    \item \url{https://linorg.usp.br/CTAN/macros/latex/contrib/abntex2/doc/abntex2cite.pdf}
    \item \url{https://linorg.usp.br/CTAN/macros/latex/contrib/abntex2/doc/abntex2cite-alf.pdf}
\end{itemize}

Ambos os pacotes contém exemplos de uso e, também, a acentuação necessária para cada entrada bibliográfica que deve ser inserida no arquivo \verb|referencias.bib|.
\cite{abnt}
%===============================
%FIM DOS ELEMENTOS TEXTUAIS (CAPS./SEÇÕES)
%Não apagar a chave a seguir.
}






%INÍCIO DOS ELEMENTOS PÓS TEXTUAIS (REFERÊNCIAS, ANEXOS, APÊNDICES)
%Início das Referências
\backmatter{
\renewcommand{\bibname}{Referências}
\bibintoc
\bibliographystyle{abntex2-alf}
\bibliography{referencias.bib}
%Fim das Referências
%Se o trabalho não tiver apêndices ou anexos
%excluir ou comentar a linha abaixo.
%APÊNDICES
\thispagestyle{empty}

\part*{Apêndices} % se quiser uma página indicativa de APÊNDICES antes dos apêndices.
%Caso não queira, comente a linha acima.
\addcontentsline{toc}{part}{Apêndices}
%Conforme a norma, os apêndices devem se organizar em ordem alfabética: A, B, C, D...
\chapterstyle{crosshead}
\chapter*{Apêndice A -- Um apêndice}
\addcontentsline{toc}{chapter}{Anexo A -- Um apêndice}

\begin{center}
	\includegraphics[scale=.60]{./img/apendice-img1.png}
\end{center}

\chapter*{Apêndice B -- Um outro apêndice}
\addcontentsline{toc}{chapter}{Anexo B -- Um outro apêndice}

\begin{center}
	\includegraphics[scale=.60]{./img/apendice-img.png}
\end{center}


%================================================
%ANEXOS
\thispagestyle{empty}
\part*{Anexos} % Se quiser uma página indicativa de ANEXOS antes dos anexos.
%Caso não queira, comente a linha acima.
\addcontentsline{toc}{part}{Anexos}
%Conforme a norma, os anexos devem se organizar em ordem alfabética: A, B, C, D...
\chapterstyle{crosshead}
\chapter*{Anexo A -- Um anexo}
\addcontentsline{toc}{chapter}{Anexo A -- Um anexo}

\begin{center}
	\includegraphics[scale=.60]{./img/anexo-img1.png}
\end{center}

\chapter*{Anexo B -- Um anexo}
\addcontentsline{toc}{chapter}{Anexo B -- Um outro anexo}

\begin{center}
	\includegraphics[scale=.60]{./img/anexo-img.png}
\end{center}



}
%FIM DOS ELEMENTOS PÓS TEXTUAIS (REFERÊNCIAS, ANEXOS, APÊNDICES)
\end{document}
