%Define o tamanho da fonte para os título das seções primárias (Large)
\chapterstyle{section}
\renewcommand*{\chapnumfont}{\normalfont\Large\bfseries}
\renewcommand*{\chaptitlefont}{\normalfont\Large\bfseries}
%Define o cabeçalho para 0pt.
\setlength\beforechapskip{-\baselineskip}
%Algarismos arábicos nas páginas
\pagenumbering{arabic}
%Estilo de página
\pagestyle{simple}
%Define a PARTE (da classe Memoir) sem numeração de páginas.
\aliaspagestyle{part}{empty}
%Define secionamento numerado até a seção quaternária.
\settocdepth{subsubsection}
\setsecnumdepth{subsubsection}
%Config. das notas de rodapé.
\setlength{\footmarkwidth}{0em}
%Espaçamento de 1pt entre os parágrafos.
\setlength{\parskip}{1\baselineskip}
%Define a indentação padrão do parágrafos.
\setlength{\parindent}{1.25cm}
%Define o espaçamento entre linhas (exceto para citações longas).
\OnehalfSpacing
%========================================================
%INÍCIO
%PACOTES ESPECÍFICOS POR ÁREA DE ESTUDO/PESQUISA

%Inclua aqui outros pacotes necessários por área de estudo.

%Incluí abaixo pacotes báscios para matemática. 
%Sugiro a inclusão de pacotes específicos para
%adequação aos objetivos do trabalho.
\usepackage{amsmath}
\usepackage{amsfonts}
\usepackage{amssymb}

%OUTROS PACOTES ESPECÍFICOS POR ÁREA DE ESTUDO/PESQUISA




%FIM

%========================================================

%======ATENÇÃO ATENÇÃO ATENÇÃO==========
%Fonte padrão do documento.
\usepackage{times}
%FONTE PADRÃO DO DOCUMENTO CONFORME
%TEMPLATE HORRÍVEL DO IFPE
%Comente \usepackage{times} acima e descomente as duas linhas abaixo
%\usepackage{helvet}
%\renewcommand{\familydefault}{\sfdefault}
%FIM DO %======ATENÇÃO ATENÇÃO ATENÇÃO=====

%Geometria do tipo de papel (A4).
\usepackage[a4paper,%
			left=3cm,%
			right=2cm,% 
			top=3cm,% 
			bottom=2cm]{geometry}
%Medir margens do pacote Geometry.
\usepackage{calc}
%Fonte 10pt para legendas de figuras e tabelas
\usepackage[font=footnotesize]{caption}
%Gerar Lorem Ipsum.
\usepackage{lipsum}
%Idiomas (o último é o principal do documento).
\usepackage[spanish,english,brazil]{babel}
%Inserir acentos .
\usepackage[utf8]{inputenc}
%Tipo de fonte usada na compilação para incluir caracteres com acentos (Ö, é, à).
\usepackage[T1]{fontenc}
%Microtipografia conforme a língua utilizada.
\usepackage[babel=true]{microtype}
%Controle e personalização de listas
\usepackage{enumitem}
\setlist{nosep}
%Hiperlinks internos e externos.
%Esconder caixa vermelha das hiperligações no documento.
%Ativar links clicáveis no pdf, com cor azul.
\usepackage[breaklinks,%
			hidelinks,%
			colorlinks=true,%
			allcolors=blue]{hyperref}
%Caso queira links com a cor igual ao texto (preto), mudar acima para 'colorlinks=false'.

%Digitar URL completas e clicáveis. 
\usepackage{xurl}
%Para escrever códigos ou trechos de códigos. Ver ambiente CODEX.
\usepackage{verbatim}
%Para escrever versos. Funciona como o verbatim.
\usepackage{alltt}
%inserir imagens.
\usepackage{graphicx}
%Inserir páginas PDF.
\usepackage{pdfpages}
%Criar multicolunas sem o ambiente "tabular".
\usepackage{multicol}
%Caixas de alerta. Útil para livros didáticos ou destaque de algum trecho importante.
\usepackage{alertmessage}
%Cores.
\usepackage{xcolor}
%Diagramas.
\usepackage{smartdiagram}
%Tabelas mais fáceis.
\usepackage{tabularx}
%Mais controle e personalização em tabelas.
\usepackage{booktabs}
%Mudar espaçamento entre linhas.
\usepackage{setspace}
%Indentação de todos os primeiros parágrafos.
\usepackage{indentfirst}
%Digitar aspas com \qq e \q.
\usepackage{textcmds}
%Facilidades para manipualar formatação de FONTES (riscado, espaçamento entre carcateres...).
\usepackage{soul}
%Destaca a primeira letra do parágrafo. Fonte maior e em caixa de texto.
\usepackage{lettrine}
%Símbolos fonéticos.
\usepackage{tipa}
%CAIXAS DE TEXTO
%Caixas de texto "quebráveis entre páginas".
\usepackage{mdframed}
\usepackage{framed}
%TCOLORBOXES
\usepackage{tcolorbox}
\usepackage{varwidth}
\tcbuselibrary{most,listingsutf8}

%Referências e Citações
\usepackage[alf,
            abnt-emphasize=bf,%
            abnt-etal-list=3,%
            abnt-etal-text=emph,%
            abnt-missing-year=sd,%
            abnt-repeated-author-omit=yes,%
            abnt-repeated-title-omit=yes,%
            abnt-thesis-year=final,%
            abnt-doi=link,%
            ]{abntex2cite}

%AMBIENTES PERSONALIZADOS
%Verbatim modificaddo para incluir códigos em listas.
\newenvironment{mverbt}
{\verbatim}%
{\endverbatim}

%Para citações longas.
\newenvironment{citel}
{\SingleSpacing\footnotesize\list{}{\rightmargin=0cm \leftmargin=4cm}%
	\item\relax}%
{\endlist}

%Texto da folha de rosto (tipo de trabalho).
\newenvironment{textofolharosto}
{\SingleSpacing\small\list{}{\rightmargin=0cm \leftmargin=7cm}%
	\item\relax}%
{\endlist}

%Ambiente de caixa: CODEX.
\newtcblisting[auto counter]{codex}[2][]{
	enhanced,
	listing only,
	title=Exemplo \thetcbcounter: #1 #2,
	listing options={style=tcblatex,%
		numbersep=1mm,%
		numbers=left,%
		numberstyle=\tiny\color{black}},
	before skip=2mm,after skip=2mm,
	colback=black!5,colframe=black!50,boxrule=0.2mm,
	attach boxed title to top left={xshift=1cm,yshift*=1mm-\tcboxedtitleheight},
	varwidth boxed title*=-3cm,
	boxed title style={frame code={
			\path[fill=tcbcolback!30!black]
			([yshift=-1mm,xshift=-1mm]frame.north west)
			arc[start angle=0,end angle=180,radius=1mm]
			([yshift=-1mm,xshift=1mm]frame.north east)
			arc[start angle=180,end angle=0,radius=1mm];
			\path[left color=tcbcolback!60!black,right color=tcbcolback!60!black,
			middle color=tcbcolback!80!black]
			([xshift=-2mm]frame.north west) -- ([xshift=2mm]frame.north east)
			[rounded corners=1mm]-- ([xshift=1mm,yshift=-1mm]frame.north east)
			-- (frame.south east) -- (frame.south west)
			-- ([xshift=-1mm,yshift=-1mm]frame.north west)
			[sharp corners]-- cycle;
		},interior engine=empty,
	},
	fonttitle=\bfseries,}
%Ambiente de caixa: OBSERVAÇÃO
\newtcolorbox{observ}[1][]{%
					arc=0mm,
					listing only,
					colback=black!5!white,
					colframe=red,
					fonttitle=\bfseries,
					title=Observação: #1,
}

%TCOLORBOX com molduras coloridas.
%Ambiente mvermelha
\colorlet{redbx}{red!75!black}
\newtcolorbox{mvermelha}[1]{%
	title={#1},
	empty,
	attach boxed title to top left,
	boxed title style={empty,size=minimal,
		toprule=2pt,top=4pt,
		overlay={\draw[redbx,line width=2pt]
			([yshift=-1pt]frame.north west)--([yshift=-1pt]frame.north east);}},
	coltitle=redbx,
	fonttitle=\Large\bfseries,
	before=\par\medskip\noindent,parbox=false,boxsep=0pt,left=0pt,right=3mm,top=4pt,
	breakable,pad at break*=0mm,vfill before first,
	overlay unbroken={\draw[redbx,line width=1pt]
		([yshift=-1pt]title.north east)--([xshift=-0.5pt,yshift=-1pt]title.north-|frame.east)
		--([xshift=-0.5pt]frame.south east)--(frame.south west); },
	overlay first={\draw[redbx,line width=1pt]
		([yshift=-1pt]title.north east)--([xshift=-0.5pt,yshift=-1pt]title.north-|frame.east)
		--([xshift=-0.5pt]frame.south east); },
	overlay middle={\draw[redbx,line width=1pt] ([xshift=-0.5pt]frame.north east)
		--([xshift=-0.5pt]frame.south east); },
	overlay last={\draw[redbx,line width=1pt] ([xshift=-0.5pt]frame.north east)
		--([xshift=-0.5pt]frame.south east)--(frame.south west);},%
}

%Ambiente mverde
\colorlet{greenbx}{green!75!black}
\newtcolorbox{mverde}[1]{%
	empty,title={#1},attach boxed title to top left,
	boxed title style={empty,size=minimal,toprule=2pt,top=4pt,
		overlay={\draw[greenbx,line width=2pt]
			([yshift=-1pt]frame.north west)--([yshift=-1pt]frame.north east);}},
	coltitle=greenbx,fonttitle=\Large\bfseries,
	before=\par\medskip\noindent,parbox=false,boxsep=0pt,left=0pt,right=3mm,top=4pt,
	breakable,pad at break*=0mm,vfill before first,
	overlay unbroken={\draw[greenbx,line width=1pt]
		([yshift=-1pt]title.north east)--([xshift=-0.5pt,yshift=-1pt]title.north-|frame.east)
		--([xshift=-0.5pt]frame.south east)--(frame.south west); },
	overlay first={\draw[greenbx,line width=1pt]
		([yshift=-1pt]title.north east)--([xshift=-0.5pt,yshift=-1pt]title.north-|frame.east)
		--([xshift=-0.5pt]frame.south east); },
	overlay middle={\draw[greenbx,line width=1pt] ([xshift=-0.5pt]frame.north east)
		--([xshift=-0.5pt]frame.south east); },
	overlay last={\draw[greenbx,line width=1pt] ([xshift=-0.5pt]frame.north east)
		--([xshift=-0.5pt]frame.south east)--(frame.south west);},%
}

%Caixas de textto com MDFRAMED
\newenvironment{bxpreta}
{\begin{mdframed}
		[skipabove=7pt,
		skipbelow=7pt,
		rightline=false,
		leftline=true,
		topline=false,
		bottomline=false,
		backgroundcolor=black!5,
		linecolor=black,
		innerleftmargin=5pt,
		innerrightmargin=5pt,
		innertopmargin=5pt,
		innerbottommargin=5pt,
		leftmargin=0cm,
		rightmargin=0cm,
		linewidth=5pt]}
	{\end{mdframed}}

\newenvironment{bxlaranja}
{\begin{mdframed}
		[skipabove=7pt,
		skipbelow=7pt,
		rightline=false,
		leftline=true,
		topline=false,
		bottomline=false,
		backgroundcolor=black!5,
		linecolor=orange,
		innerleftmargin=5pt,
		innerrightmargin=5pt,
		innertopmargin=5pt,
		innerbottommargin=5pt,
		leftmargin=0cm,
		rightmargin=0cm,
		linewidth=5pt]}
	{\end{mdframed}}

%Caixa Fundo cinza, Borda cinza
\newenvironment{bxcinza}
{\begin{mdframed}
		[skipabove=7pt,
		skipbelow=7pt,
		rightline=false,
		leftline=true,
		topline=false,
		bottomline=false,
		linecolor=gray,
		backgroundcolor=black!5,
		innerleftmargin=5pt,
		innerrightmargin=5pt,
		innertopmargin=5pt,
		leftmargin=0cm,
		rightmargin=0cm,
		linewidth=4pt,
		innerbottommargin=5pt]}
	{\end{mdframed}}

%Caixa Fundo azul, Borda azul
\newenvironment{bxazul}
{\begin{mdframed}
		[skipabove=7pt,
		skipbelow=7pt,
		rightline=false,
		leftline=true,
		topline=false,
		bottomline=false,
		linecolor=blue!40!gray,
		backgroundcolor=black!5,
		innerleftmargin=5pt,
		innerrightmargin=5pt,
		innertopmargin=5pt,
		leftmargin=0cm,
		rightmargin=0cm,
		linewidth=4pt,
		innerbottommargin=5pt]}
	{\end{mdframed}}

%Borda vermelha
\newenvironment{bxvermelha}
{\begin{mdframed}
		[skipabove=7pt,
		skipbelow=7pt,
		rightline=false,
		leftline=true,
		topline=false,
		bottomline=false,
		backgroundcolor=black!5,
		linecolor=red!40!gray,
		innerleftmargin=5pt,
		innerrightmargin=5pt,
		innertopmargin=5pt,
		innerbottommargin=5pt,
		leftmargin=0cm,
		rightmargin=0cm,
		linewidth=5pt]}
	{\end{mdframed}}